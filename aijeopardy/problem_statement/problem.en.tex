\problemname{AI Jeopardy}

\illustration{0.31}{sanbot}{\href{https://en.wikipedia.org/wiki/File:Sanbot_King_Kong.jpg}{Picture} by QIHAN Technology}%
\noindent
The robot revolution is finally here, albeit not quite in the highly
explosive way envisioned in various science fiction books and movies.
It seems that, perhaps due to a small typo in the AI source code, the
robots are not taking our lives but instead taking our livelihoods.
One early job market fatality in this revolution was the (somewhat
niche) occupation as jeopardy player: already in 2011 the
\emph{Watson} computer defeated two legendary but inferior human
jeopardy champions.

Nowadays, more and more of Jeopardy's
\emph{viewers} are AIs themselves and as such the show is considering
having categories on topics that are more popular with this new
number-crunching viewer base.  Focus group testing has revealed that
AIs are particularly fond of the ``Binomial Coefficients'' category.  The
premise of this category is that the answer that contestants get is
some positive integer $X$, and the contestants must respond with a
question of the form ``What is $n$ choose $k$?'' (and this is a
correct response if the binomial coefficient $n$ choose $k$ equals $X$).

Write an AI to play this new Jeopardy category.  If there are several
different possible solutions for $n$ and $k$, the AI should choose the
most elegant solution, having the smallest value of $n$, and of those
with the smallest $n$ it should choose the one with the smallest value
of $k$.

\section*{Input}

Input consists of a single integer $X$ ($1 \le X \le 10^{100}$).

\section*{Output}

Output two non-negative integers $n$ and $k$ such that the binomial coefficient $n$
choose $k$ equals $X$, with ties between multiple solutions broken as
explained above.
