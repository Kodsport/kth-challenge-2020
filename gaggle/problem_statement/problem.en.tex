\problemname{Gaggle}

\illustration{0.47}{formation}{\href{https://pixabay.com/photos/formation-migratory-birds-geese-508038/}{Picture} by Antranias}%
\noindent
At the new start-up company Gaggle, we have rejected the oppressive
corporate structures of old, with all of their managers and subordinates
and hierarchies and so on. Instead we have embraced a free and open
corporate culture in which all employees (called Gagglers) are in
charge of themselves and allowed to roam free.

Rather than having managers overseeing the work, the main method used
 to coordinate work at Gaggle is a mentor system: each Gaggler
designates some other Gaggler as their mentor, with whom they discuss
their ongoing projects.  This mentor relation may or may not be
symmetric (in other words you may or may not be the mentor of your
mentor) but you can never be the mentor of yourself.

Initially, all Gagglers were able to pick anyone they liked as their
mentor, but after a while it was discovered that this lead to two
problems:

\begin{enumerate}
\item Some people were more popular than others and had too many choosing
them as their mentor, causing them not to have time to do their actual
work.

\item Some flocks of Gagglers ended up isolated from the rest of the
company (e.g.,~if Gagglers $A$ and $B$ are each other's mentors and they
are not the mentor of anyone else), causing failure of these flocks to
coordinate with the rest of the company.
\end{enumerate}

In order to remedy these two flaws, it was (collectively) decided
that:

\begin{enumerate}
\item Every Gaggler must be the mentor of exactly one other Gaggler, and

\item Assuming every Gaggler only communicates with their mentor and their mentee, it must still be possible for any information that any Gaggler has to reach any other Gaggler.
\end{enumerate}

In order to reward lower-numbered (more senior) Gagglers while
introducing this new policy, it was decided that lower-numbered
Gagglers should get to keep their current mentor if possible, and if
they have to change, their new mentor should be as low-numbered (more
senior, and therefore more experienced) as possible.

Concretely, consider two possible new assignments of mentors, and
suppose the lowest-numbered Gaggler where these assignments differ is
Gaggler number $i$.  Then if one of the two assignments assigns
Gaggler $i$ the same mentor as they originally had, we prefer that
assignment.  Otherwise, if Gaggler $i$ gets a new mentor in both of
the two assignments, then we prefer the assignment where the number of
the new mentor of Gaggler $i$ is smaller.

For example, consider Sample Input 2 below.  One possible new
assignment of mentors would be to simply change so that Gaggler $1$
becomes mentored by Gaggler $2$.  However, in the best assignment,
shown in Sample Output 2, we let Gaggler $1$ keep their current mentor
and instead change the mentors of both Gagglers $2$ and $3$.

\section*{Input}

The first line of input contains a single integer $n$ ($2 \le n \le
500\,000$), the number of Gagglers.  Then follows a line containing
$n$ integers $a_1, a_2, \ldots, a_n$ ($1 \le a_i \le n$ and $a_i \ne
i$ for each $i$) where $a_i$ is the current mentor of Gaggler $i$ (the
Gagglers are numbered from $1$ to $n$).

\section*{Output}

Then output a line with the new assignment $b_1, \ldots, b_n$ of
mentors, in the same format as in the input.  The new list should be a
valid assignment according to the new requirements, and be the best
according to the tie-breaking rule described above.
